% Options for packages loaded elsewhere
\PassOptionsToPackage{unicode}{hyperref}
\PassOptionsToPackage{hyphens}{url}
%
\documentclass[
]{book}
\usepackage{lmodern}
\usepackage{amssymb,amsmath}
\usepackage{ifxetex,ifluatex}
\ifnum 0\ifxetex 1\fi\ifluatex 1\fi=0 % if pdftex
  \usepackage[T1]{fontenc}
  \usepackage[utf8]{inputenc}
  \usepackage{textcomp} % provide euro and other symbols
\else % if luatex or xetex
  \usepackage{unicode-math}
  \defaultfontfeatures{Scale=MatchLowercase}
  \defaultfontfeatures[\rmfamily]{Ligatures=TeX,Scale=1}
\fi
% Use upquote if available, for straight quotes in verbatim environments
\IfFileExists{upquote.sty}{\usepackage{upquote}}{}
\IfFileExists{microtype.sty}{% use microtype if available
  \usepackage[]{microtype}
  \UseMicrotypeSet[protrusion]{basicmath} % disable protrusion for tt fonts
}{}
\makeatletter
\@ifundefined{KOMAClassName}{% if non-KOMA class
  \IfFileExists{parskip.sty}{%
    \usepackage{parskip}
  }{% else
    \setlength{\parindent}{0pt}
    \setlength{\parskip}{6pt plus 2pt minus 1pt}}
}{% if KOMA class
  \KOMAoptions{parskip=half}}
\makeatother
\usepackage{xcolor}
\IfFileExists{xurl.sty}{\usepackage{xurl}}{} % add URL line breaks if available
\IfFileExists{bookmark.sty}{\usepackage{bookmark}}{\usepackage{hyperref}}
\hypersetup{
  pdftitle={Meyer Lab Handbook},
  pdfauthor={Jesse Meyer},
  hidelinks,
  pdfcreator={LaTeX via pandoc}}
\urlstyle{same} % disable monospaced font for URLs
\usepackage{longtable,booktabs}
% Correct order of tables after \paragraph or \subparagraph
\usepackage{etoolbox}
\makeatletter
\patchcmd\longtable{\par}{\if@noskipsec\mbox{}\fi\par}{}{}
\makeatother
% Allow footnotes in longtable head/foot
\IfFileExists{footnotehyper.sty}{\usepackage{footnotehyper}}{\usepackage{footnote}}
\makesavenoteenv{longtable}
\usepackage{graphicx,grffile}
\makeatletter
\def\maxwidth{\ifdim\Gin@nat@width>\linewidth\linewidth\else\Gin@nat@width\fi}
\def\maxheight{\ifdim\Gin@nat@height>\textheight\textheight\else\Gin@nat@height\fi}
\makeatother
% Scale images if necessary, so that they will not overflow the page
% margins by default, and it is still possible to overwrite the defaults
% using explicit options in \includegraphics[width, height, ...]{}
\setkeys{Gin}{width=\maxwidth,height=\maxheight,keepaspectratio}
% Set default figure placement to htbp
\makeatletter
\def\fps@figure{htbp}
\makeatother
\setlength{\emergencystretch}{3em} % prevent overfull lines
\providecommand{\tightlist}{%
  \setlength{\itemsep}{0pt}\setlength{\parskip}{0pt}}
\setcounter{secnumdepth}{5}
\usepackage{booktabs}
\usepackage{amsthm}
\makeatletter
\def\thm@space@setup{%
  \thm@preskip=8pt plus 2pt minus 4pt
  \thm@postskip=\thm@preskip
}
\makeatother
\usepackage[]{natbib}
\bibliographystyle{apalike}

\title{Meyer Lab Handbook}
\author{Jesse Meyer}
\date{2020-10-21}

\begin{document}
\maketitle

{
\setcounter{tocdepth}{1}
\tableofcontents
}
\hypertarget{welcome}{%
\chapter{Welcome!}\label{welcome}}

Welcome to the Meyer Lab Handbook.
\url{https://jessegmeyerlab.github.io/Handbook/}

The handbook was developed by \href{https://www.jessemeyerlab.com/our-team}{Jesse Meyer} as a guide for current and prospective researchers with resources, guidelines, and information about the Meyer Lab.

More information on the lab and department can be found through the following resources:

\begin{itemize}
\tightlist
\item
  \href{https://www.jessemeyerlab.com}{The Lab website}
\item
  \href{https://www.mcw.edu/departments/biochemistry}{MCW Biochemistry Departmental website}
\end{itemize}

This handbook was forked from \href{https://www.schweppelab.org/}{Devin Schweppe's} \href{https://schweppelab.github.io/Handbook/index.html}{handbook}, which was adapted from and influenced by several other works from the \href{https://ccmorey.github.io/labHandbook/index.html}{Morey}, \href{https://github.com/jpeelle/peellelab_manual/blob/master/peellelab_manual.pdf}{Peele}, and \href{http://www.thememolab.org/resources/}{Ritchey} groups. The Meyer Lab manual is licensed under the \href{https://creativecommons.org/licenses/by/4.0/}{Creative Commons Attribution 4.0 International License}.

\hypertarget{intro}{%
\chapter{Introduction}\label{intro}}

\hypertarget{lab-member-expectations-and-responsibilities}{%
\section{Lab Member Expectations and Responsibilities}\label{lab-member-expectations-and-responsibilities}}

If you work hard for me, I'll work hard for you.

\hypertarget{for-everyone}{%
\subsection{For Everyone}\label{for-everyone}}

The following applies to all lab members with any status.

13 Lab Laws.

\emph{Be:}

\begin{itemize}
\tightlist
\item
  \textbf{Trustworthy.} We all depend on each other. Keep your promises.
\item
  \textbf{Loyal.} If you work hard for me, I will work hard for you.
\item
  \textbf{Helpful.} We all need help sometimes. Offer to help without expecting any reward.
\item
  \textbf{Friendly.} Approach everyone as if they are your long-time friend.
\item
  \textbf{Courteous.} We always must show respect and consideration.\\
\item
  \textbf{Kind.} Treat others the way you want to be treated. Primum non nocere
\item
  \textbf{Obedient.} If I ask for your help with something, I expect you to do your best to help. This goes both ways. If you ask me for something, I will do my best to do it.
\item
  \textbf{Cheerful.} We are privileged to be where we are, and that should make you cheerful. Look for the positive side of every situation!
\item
  \textbf{Thrifty.} The public entrusts their tax dollars to support our research, and in exchange we promise to be careful with that money.
\item
  \textbf{Brave.} You will need to do things you're not comfortable with. Maybe that means an experiment, conference, or public speaking. From experience believe me that 90\% of success is showing up, and you will be stronger after.
\item
  \textbf{Clean.} Please keep common areas clean and without clutter. Items left unattended may be cleaned, reclaimed, or recycled. If you're using lab equipment, put it away when you're done.
\item
  \textbf{Reverant.} Be reverent towards science and your journey.
\item
  \textbf{Expedient.} Plan full days of research. Act like someone might finish and publish your project before you, because in reality, they might.
\end{itemize}

\textbf{\emph{Details}}

\begin{itemize}
\tightlist
\item
  Do not come into the lab if you are sick. Stay home and get healthy, and don't risk getting others sick.
\item
  Notify the lab manager or me if you will be taking the day off from work, either due to illness or vacation. If you are sick and you had experiments or meetings scheduled that day, notify your participants or collaborators and reschedule.
\item
  \emph{No food} is allowed at desks in the bays. Food can be consumed in the dry lab area, the common areas outside of the lab, or outside on a nice day!
\item
  You are not expected to work on staff holidays. You are expected to work during university breaks (except for staff holidays or if you're taking your paid vacation/personal time).
\item
  Lock the doors to the lab if no one else is around, even if you're stepping out for a minute.
\item
  The dress code is casual for routine work days. The only requirements in the lab are long pants and closed-toed shoes. Please dress semi-professional when presenting or interviewing new potential lab mates.\\
\item
  When working remotely, you should be generally available during workdays (not necessarily responding immediately, but ideally within a few hours), and you should attend any scheduled remote lab meetings.
\item
  In general, reply to emails you recieve (not only from me) within 24 hours.
\item
  Notify me if you must miss or reschedule our group or one-on-one meetings
\end{itemize}

\hypertarget{for-the-pi}{%
\subsection{For the PI}\label{for-the-pi}}

All of the points in the \emph{Everyone} section, and you can expect me to:

\begin{itemize}
\tightlist
\item
  Guide our group's collective research vision
\item
  Secure the funding you need for salary and supplies
\item
  Meet with you regularly to discuss your research progress. Meeting frequency will change over time or over the course of a project. I will meet with new lab members for at least 15 minutes every morning, but as you progress meetings may be less frequent.
\item
  Develop a mentoring and research plan tailored to your interests, needs, and career goals. We will meet in September each year to discuss your strategic plan for the academic year to keep you on track with your goals, and we will meet in March to review progress toward these goals.
\item
  Guide your professional development, for example helping with presentations and writing
\item
  Support your career development by introducing you to other researchers in the field, writing recommendation letters for you, providing you with opportunities to attend at least one conference per year, and promoting your work in my own presentations.
\item
  Care about you as a person and not just a scientist. I am happy to discuss with you any concerns or life circumstances that may be influencing your work, but it is entirely up to you whether and what you want to share.
\item
  If you need extra support related to time management and productivity, I will brainstorm solutions with you and share what has worked for me and for others.
\end{itemize}

\hypertarget{for-post-docs}{%
\subsection{For Post-docs}\label{for-post-docs}}

All of the points in the \emph{Everyone} section, and they are expected to:

\begin{itemize}
\tightlist
\item
  Develop your own independent line of research.
\item
  Mentor undergraduate and graduate students on their research projects, when asked or when appropriate.
\item
  Apply for external funding (e.g., NRSA, K99, Damon-Runyon, Ford). I will hire postdocs only when there is funding available for at least a year; however, applying for external funding is a valuable experience and, if awarded, it will release those dedicated funds for other purposes.
\item
  Apply for jobs (academic or industry or otherwise) as soon as you are ``ready'' and/or by the beginning of your fourth year as a postdoc.
\item
  If you are planning to pursue a non-academic career, treat your postdoctoral research as seriously as you might if you were pursuing an academic career. Excellent research productivity, demonstrated by publications, is beneficial for every path.
\item
  Remind me (the PI) that different scientific opinions can co-exist in the same lab!
\item
  Postdocs must involve the PI in any manuscripts they are working on (excluding those from previous labs). The PI may refuse to participate intellectually or as author.
\end{itemize}

\hypertarget{for-graduate-students}{%
\subsection{For Graduate Students}\label{for-graduate-students}}

All of the points in the \emph{Everyone} section, and they are expected to:

\begin{itemize}
\tightlist
\item
  Publication requirements: all PhD students are expected to publish \emph{at least} one first-author review paper and two first-author research papers. Exceptions to the research can be made for one high impact article (Cell, Nature, Science).
\item
  Start writing a review on a topic we agree upon the day you decide to join the lab. The purpose of this is to both bolster your CV, and to immerse you into your field so we can see the gaps where research will have the most impact.
\item
  Develop a line of dissertation research. Ideally, your dissertation research will consist of at least 3 related experiments that produce at least 2 manuscripts.
\item
  Apply for external funding (e.g., NSF GRFP or NRSA). This is a valuable learning experience and a great honor if awarded.
\item
  Work toward knowing what you want after you graduate and communicate those thoughts to me. I can help you go anywhere, but I need to know where you want to go.
\item
  Be responsible for your own academic deadlines and keep me updated on them. You must know when department requirements are due (advancement to candidacy etc). In general, this includes your external funding applications, your qualifying exam, and your dissertation proposal and defense.
\item
  Prioritize time for research. It is easy to get caught up in coursework or TA-ing, but at the end of 5-ish years, you need to have completed a dissertation.
\item
  Students must involve the PI in any manuscripts they are working on (excluding those from previous labs). The PI may refuse to participate intellectually or as author.
\end{itemize}

\hypertarget{for-lab-managers-research-scientists}{%
\subsection{For Lab Managers \& Research Scientists}\label{for-lab-managers-research-scientists}}

All of the points in the \emph{Everyone} section, and they are expected to:

\begin{itemize}
\tightlist
\item
  Maintain the lab IRB protocols and paperwork (e.g., archiving consent forms).
\item
  Oversee hiring, scheduling, training and ordering.
\item
  Maintain the lab internal website.
\item
  Keep the lab manager manual up to date.
\item
  Assist with participant recruitment and scheduling.
\item
  Assist other lab members with data collection or analysis (typically you will be assigned to particular projects).
\item
  Coordinate and take notes during weekly lab meetings.
\item
  Help to maintain an atmosphere of professionalism within the lab.
\item
  Work on your own research project.
\end{itemize}

\hypertarget{data-and-reproducibility}{%
\subsection{Data and reproducibility}\label{data-and-reproducibility}}

\begin{itemize}
\tightlist
\item
  All data must be uploaded to a public repository before publication.
\item
  All code needed to reproduce a result must be shared publicly.
\end{itemize}

\hypertarget{lab-architecture}{%
\section{Lab Architecture}\label{lab-architecture}}

The PI, for better or worse, shoulders responsibility for the work conducted by their lab group. While everyone involved in the work will be acknowledged when work we have done is published or praised, the PI will always be primarily responsible for correcting problems when they arise, no matter who really caused them. Our work can be questioned years after it has been carried out and published, meaning the PI is the only person committed to this for long enough to realistically keep this commitment.

For some post-doctoral projects, the researchers involved might share a long-term commitment to the research and be the ``local PI'' on the work. In those cases, they will act as the primary person responsible for those projects. Even so, the PI must always have access to enough information about these projects to independently reproduce analyses and replicate findings.

While the PI thinks in terms of large, multi-experiment projects, lab researchers at all levels will have the responsibility for individual experiments, projects, or component projects. Elements of any project must always be documented. Every project has designated milestones at which documentation should be completed, backed-up, and shared (at least with the lab group, often publicly).

Whenever a lab member moves on from the lab, every project they led must be documented and made accessible to the PI. At the point of leaving the lab, if a project is unpublished, the PI must be given full editing authority along with the former lab member.

\hypertarget{current-lab-members}{%
\section{Current Lab Members}\label{current-lab-members}}

Jesse G. Meyer, PhD, Principal Investigator (PI)

\hypertarget{postdoctoral-fellows}{%
\subsection{Postdoctoral Fellows}\label{postdoctoral-fellows}}

\hypertarget{graduate-students}{%
\subsection{Graduate Students}\label{graduate-students}}

\hypertarget{research-scientists}{%
\subsection{Research Scientists}\label{research-scientists}}

Quinn Dickinson, MS, Research Technologist I

\hypertarget{departmental-resources}{%
\section{Departmental Resources}\label{departmental-resources}}

Additional resources will go here

\hypertarget{code}{%
\chapter{Code of Conduct}\label{code}}

\hypertarget{general}{%
\section{General}\label{general}}

In addition to the general expectations laid out above, I am dedicated to making our lab a safe, inclusive, and welcoming environment for all. Below you can find a specific code of conduct for behavior in the lab, as well as a broader discussion of what constitutes an inclusive environment. For more information on professional conduct see the \href{https://www.mcw.edu/about-mcw/non-discrimination-notice/mcw-professional-conduct-policy}{MCW Policy on Professional Conduct}.

Please visit the Medical College of Wisconsin's website for the department of \href{https://www.mcw.edu/departments/office-of-diversity-and-inclusion}{diversity and inclusion}.

\hypertarget{building-an-inclusive-lab-environment}{%
\section{Building an Inclusive Lab Environment}\label{building-an-inclusive-lab-environment}}

All members of the lab, along with visitors, are expected to agree to the following code of conduct. More information and training on respect in the workplace is available from MCW's \href{https://www1.mcw.edu/FileLibrary/Groups/FacultyAffairs/NewFacultyForms/NewFaculty-FacultyRespectTrain3.pdf}{website}.

\hypertarget{code-of-conduct}{%
\subsection{Code of Conduct}\label{code-of-conduct}}

The lab is dedicated to providing a harassment-free experience for everyone, regardless of gender, gender identity and expression, age, sexual orientation, disability, socioeconomic status, physical appearance, body size, race, national origin, or religion (or lack thereof). We do not tolerate harassment of lab members in any form.

All lab members will treat one another with respect and be sensitive to how one's words and actions impact others. We do not tolerate the perpetuation of stereotypes; we do not tolerate other acts of microaggression (\href{https://www.washington.edu/teaching/topics/inclusive-teaching/addressing-microaggressions-in-the-classroom/}{more information}). We are a team. We stand up for one another. We learn from each other. We hold each other accountable.

\hypertarget{scientific-integrity}{%
\section{Scientific Integrity}\label{scientific-integrity}}

\hypertarget{reproducible-research}{%
\subsection{Reproducible Research}\label{reproducible-research}}

I expect that all of our research will be, at minimum, reproducible (when possible, we will also test for replicability). As a researcher, it is your responsibility to ensure reproducibility of your research by: 1) detailed note-taking and 2) programming workflows with version control.

Programming workflows help with reproducibility because they take away choice and therefore variability. The ideal scenario is that we have a script or series of scripts that takes data from raw form to final product. Programming alone is not enough, though, because people can easily forget which script changes they made and when. Therefore, all projects that involve programming of any kind (so basically, all projects) must use some form of version control. I strongly recommend git in combination with GitHub (see below), unless you have a pre-existing workflow.

\hypertarget{authorship}{%
\subsection{Authorship}\label{authorship}}

Authorship will be discussed prior to the beginning of a new project, so that expectations are clearly defined. If you feel that authorship was not adequately discussed, please raise your concerns and questions early and often. These decisions are made more difficult at the final stage of a project when we are ready to publish. However, changes to authorship may occur over the course of a project if a new person becomes involved or if someone is not fulfilling their planned role. In most cases, graduate students and postdocs will be a first author on publications on which they are the primary lead, and I will be a last/corresponding author.

\hypertarget{old-projects}{%
\subsection{Old Projects}\label{old-projects}}

For projects that required significant lab resources (e.g.~large scale proteomics experiments): Project ``ownership'' expires 12 months after data collection has ended or whenever the original primary lead relinquishes their rights to the study, whichever comes first. At that point, I reserve the right to re-assign the project (or not) as needed to expedite publication. This policy is intended to avoid situations in which a dataset languishes for a long period of time, while still giving publication priority to the original primary lead. I will never exclude you from participating, but decisions to change the lead are final and must be accepted.

\hypertarget{general}{%
\chapter{General Policies}\label{general}}

\hypertarget{pi-availability}{%
\section{PI Availability}\label{pi-availability}}

Until the pandemic ends, I will work from home a significant proportion of the time. After the pandemic, I expect to have times where my door is open, at which times you should feel free to come talk with me. Unless there is an emergency, if my door is closed, please send me an email or try back later instead of knocking.

When working remotely, I am available over Teams or for ad-hoc meetings during regular office hours.

\hypertarget{vacation}{%
\section{Vacation}\label{vacation}}

Everyone is allowed 15 days of vacation per year (3 work weeks). Please let me know at least 2 weeks in advance of your scheduled vacations.

\hypertarget{meetings}{%
\section{Meetings}\label{meetings}}

\hypertarget{lab-meetings-journal-club}{%
\subsection{Lab Meetings \& Journal Club}\label{lab-meetings-journal-club}}

Weekly lab meetings will be focused on project presentations and going over new data/methods. Lab meetings will last 1.5 hours. If at the end of 1.5 hours, we need more time to discuss something, we will schedule another meeting. \textbf{All full-time lab members} including rotation students are expected to attend the weekly lab meeting. All part-time lab members (including undergraduates) are welcome to attend but attendance is not required.

In addition, we sometimes hold journal clubs in addition to lab meetings with research presentations. Journal clubs will focus on discussing new and/or important research articles. Some weeks, we'll discuss a single article that everyone has read; other weeks, we'll each read a paper on a specific theme and do mini-presentations on each paper. As with our internal lab meetings, all full-time lab members are expected to attend these additional meetings, and part-time lab members are invited but not required to attend.

During extended periods of working remotely, such as during the COVID-19 pandemic, we will also have regular lab ``check-ins'' (currently on a Tuesday-Thursday schedule) to set up our goals for the week, encourage casual interactions, etc.

\hypertarget{individual-meetings}{%
\subsection{Individual Meetings}\label{individual-meetings}}

At the beginning of each semester, I will set a schedule to meet with each full-time lab member for one hour a week. If we do not have anything to discuss in a given week, that's is OK, we can just say hi or cancel it. Before each meeting, update your meeting agenda; this will also be a place where we document next steps. Over the summer, we may set the schedule on a weekly basis since summer schedules are more flexible and variable.

\hypertarget{joint-lab-meetings}{%
\subsection{Joint Lab Meetings}\label{joint-lab-meetings}}

Occasionally we will participate in joint lab meetings, or join other labs for their lab meeting. As with our own meetings, be respectful, be supportive, and \emph{be on time} at these meetings. The GS department is full of great colleagues and these meetings are an important opportunity for collaborative thinking and projects.

\hypertarget{work-hours}{%
\section{Work Hours}\label{work-hours}}

One of the benefits of a career in academic research is that it is typically more flexible than other kinds of jobs. However, you should still treat it like a job. If you are employed for 40 hours a week, you should be working 40 hours a week. This applies to lab staff members and postdocs. You are not required to work over-time. For graduate students, I recognize that you have other demands on your time like classes and TA-ing but I still expect that you will be regularly engaged in your research.

Lab staff members are expected to keep regular hours (e.g., somewhere in the ballpark of 9-5). Graduate students and postdocs have more flexibility. However, in order to encourage lab interaction, I expect that all lab members will be in the lab (or available on Slack, when working remotely), at minimum, most weekdays between 11am and 4pm or so. If you're going to be taking off from work on a normal workday (i.e., taking vacation or a personal or sick day), please let me know.

\hypertarget{deadlines}{%
\section{Deadlines}\label{deadlines}}

If you need something from me by a particular deadline, please inform me as soon as you are aware of the deadline so that I can allocate my time as efficiently as possible. I will expect at least one week's notice, but I greatly prefer two weeks' notice. I will require two weeks' notice for letters of reference. If you do not adhere to these guidelines, I may not be able to meet your deadline. Please note that this applies to reading/ commenting on abstracts, papers, and manuscripts, in addition to filling out paperwork, etc. Reminder messages are appreciated as well!

\hypertarget{presentations}{%
\section{Presentations}\label{presentations}}

I encourage you to seek out opportunities to present your research to the department, research community, or general public. If you are going to give a presentation (including posters and talks), be prepared to give a practice presentation to the lab at least one week ahead of time. Not only will this help you feel comfortable with the presentation, it will give you time to implement any feedback. I care about practice presentations because \textbf{a)} presenting your work is a huge part of being successful in science and it is important that you practice those skills as often as possible, and \textbf{b)} you are going to be representing not only yourself but also the rest of the lab.

There is a lab template for posters that you are free to modify as you see fit, but the header and general aesthetic should stay similar. If you have ideas for how to improve the poster template, please show the lab so we can decide whether to implement them as a group. This will help increase the visibility of our lab at conferences. There is no template for talks, and I encourage you to use your own style of presentation as long as it is polished and clear.

When making figures, it is helpful if you follow a few color-coding conventions, so that it's easier to keep things consistent when I present your work in talks. For example, please use the viridis color scheme for heatmaps and any color mapping to esure visibility to color blind. There are also common figures and illustrations available.

\hypertarget{lab-travel}{%
\section{Lab Travel}\label{lab-travel}}

The lab will typically pay for full-time lab members to present their work at major conferences (e.g.~ASMS, Keystone, HuPO). In general, the work should be ``new'' in that it has not been presented previously, and it should be appropriate for the conference. I expect to support your attendance to one conference per year. Meal costs and lodging will be reimbursed according to the MCW policy. The lab will also pay for new grad students and postdocs to attend one conference in their first year in lab (i.e.~without presenting). If you wish to attend any other conference outside of these guidelines, come speak with me. If travel expenses are being paid off of a grant, additional restrictions may apply (come speak with me). These guidelines may change depending on the availability of funds. I recommend that lab members apply for other sources of funding available to them (e.g.~departmental funds for grad students, ASMS travel awards).

\hypertarget{letters-of-referencerecommendation}{%
\section{Letters of Reference/Recommendation}\label{letters-of-referencerecommendation}}

Letters of reference/recommendation are one of the many benefits of working in a research lab. I will write a letter for any student or lab member who has spent at least one year in the lab. Letters will be provided for shorter-term lab members in exceptional circumstances (e.g.~new graduate students or postdocs applying for fellowships). I maintain this policy because I do not think that I can adequately evaluate someone who has been around for less than a year.
To request a letter of recommendation, please adhere to the deadline requirements described above. Send me your current CV and any relevant instructions for the contents of the letter. If you are applying for a grant, send me your specific aims or a short summary of the grant. In some but not all cases, I may ask you to draft a letter, which I will then revise to be consistent with my evaluation. This will ensure that I do not miss any details about your work that you think are relevant to the position you're applying for, and it will also help me complete the letter in a timely fashion. Please feel free to send me a reminder email close to the letter deadline.

\hypertarget{funding}{%
\section{Funding}\label{funding}}

Funding for the lab comes from a variety of sources, including the lab startup fund, federal agencies (e.g.~NIH), private foundations, and internal funds from the University of Washington. I will oversee all aspects of the financial management of our funding sources. However, it is important to me to be transparent about where research money comes from and how it's spent. Please ask if you want to know more details. In general, external funds tend to be restricted to expenses related to a particular project or set of projects, whereas some of the internal funds are flexible in that they can be used for any justifiable work-related purpose.

All research funded by external grants must acknowledge the funding agency and grant number upon publication.

\hypertarget{safety}{%
\section{Safety}\label{safety}}

Part of maintaining a great working environment in the lab is maintaining safe working habits. For more information on safety in the lab check out these resources:

\begin{itemize}
\tightlist
\item
  \href{https://www.mcw.edu/departments/research-systems/training/safety}{EHS Trainings}
\item
  \href{https://www.mcw.edu/departments/institutional-biosafety-committee-ibc/biosafety-levels}{Biosafety levels}
\item
  \href{https://infoscope.mcw.edu/FileLibrary/Groups/InfoScopeSafety/Manuals/SafetyManual_2017.pdf}{MCW safety manual}
\end{itemize}

\hypertarget{data}{%
\chapter{Data and Instrumentation}\label{data}}

\hypertarget{instrumentation}{%
\section{Instrumentation}\label{instrumentation}}

We work with very expensive instrumentation that requires very expensive reagents and supplies. This secion give guidlines for the use of that instrumentation.

Instrumentation and equipment need to be maintained so that every member of the lab has access to complete their research projects. When you have finished using an instrument, you should leave it in a condition so that the next person who needs to use it does not need to fix/clean it. This essential to keeping the lab functioning efficiently.

\hypertarget{mass-spectrometers}{%
\subsection{Mass spectrometers}\label{mass-spectrometers}}

Mass spectrometers (MSs) are the key technology we use to drive projects. Over the course of your work in the lab there will be opportunities to learn more about using, running and fixing the the MSs. I highly encourage you to learn as much as you can as well through the literature to understand what the instruments are capable of and not capable of.

If you run into issues in the operation of the instruments (including but not limited to hardware malfunctions or failures and software faults or errors), \textbf{alert me immediately}! These problems cannot be ignored and are more likely than not to cause considerable harm to the instrument if left unattended. Unaddressed issues are also likely to result in expensive instrument repairs (tens of thousands of dollars).

If you notice significant degradation of instrument performance during or after your runs, alert both Jesse and the next user so that samples are not lost!

\textbf{Before and after every scheduled use} you are requred to analyze a standard that captures the performance of the method you are using. You are responsible for returing the instrument to the next user in the same condition as you recieved it, and comparing the standard data before and after your analysis proves this. If you are doing nanoLC-MS/MS, then that standard should show the separation AND mass spectral quality and sensitivity. I will work with each of you to determine the best standard for your project.

\hypertarget{hplcs}{%
\subsection{HPLCs}\label{hplcs}}

Separation or enrichment of peptides, proteins, and small molecules using high performance liquid chromatography (HPLC) is an integral part of our workflows. HPLC systems are usually the main source of instrument problems. As with the mass spectrometers any issues with the hardware or software need to be addressed immediately.

\hypertarget{data-data-data}{%
\section{Data, Data, Data}\label{data-data-data}}

\hypertarget{scripts-and-code}{%
\subsection{Scripts and Code}\label{scripts-and-code}}

All scripts and code used for lab projects (including programs, websites, tools, data analysis work) should be deposited or version controlled in lab storage/servers. This includes depositing repositories in the lab GitHub.

\hypertarget{raw-data}{%
\subsection{Raw Data}\label{raw-data}}

Raw data will be backed up in at least two places to ensure that should it need to be accessed it can be. This is especially important for published data as someone may ask for these files years later. Unpublished data should be similarly managed to ensure that we do not need to revisit the same basic work again. If you are unsure or where to store your data contact Jesse!

\textbf{AS SOON AS YOU FINISH COLLECTING DATA FOR YOUR PROJECT} it should be uploaded to massive or a similar repository with metadata that describes the file naming convention.

\hypertarget{data-sharing}{%
\subsection{Data sharing}\label{data-sharing}}

Data for publication will be shared through PRIDE/ProteomeXchange or a similar repository. Sharing these data with the general community is essential for future development of MS methods.

\hypertarget{lab-storage}{%
\subsection{Lab Storage}\label{lab-storage}}

We have a lab drive setup through the MCW research computing center. All data should be copied from the mass spectrometer as soon as it is collected and stored on our group drive.

\hypertarget{literature}{%
\chapter{Literature}\label{literature}}

Background literature for the lab can be found in the ``Getting Started'' Google Drive folder. You will be given access to this upon joining. If you need access, ask Devin about this.

Below find links to papers that may be of interest to explore new topics or get caught up on where the lab is now.

\hypertarget{proteomics}{%
\section{Proteomics}\label{proteomics}}

\hypertarget{proteome-data-analysis}{%
\section{Proteome data analysis}\label{proteome-data-analysis}}

\hypertarget{deep-learning}{%
\section{Deep learning}\label{deep-learning}}

\hypertarget{basic-mass-spectrometry}{%
\section{Basic mass spectrometry}\label{basic-mass-spectrometry}}

  \bibliography{book.bib,packages.bib}

\end{document}
